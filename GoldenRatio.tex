%%%%%%%%%%%%  Generated using docx2latex.com  %%%%%%%%%%%%%%

%%%%%%%%%%%%  v2.0.0-beta  %%%%%%%%%%%%%%

\documentclass[12pt]{article}
\usepackage{amsmath}
\usepackage{latexsym}
\usepackage{amsfonts}
\usepackage[normalem]{ulem}
\usepackage{array}
\usepackage{amssymb}
\usepackage{graphicx}
\usepackage[backend=biber,
style=numeric,
sorting=none,
isbn=false,
doi=false,
url=false,
]{biblatex}\addbibresource{bibliography.bib}

\usepackage{subfig}
\usepackage{wrapfig}
\usepackage{wasysym}
\usepackage{enumitem}
\usepackage{adjustbox}
\usepackage{ragged2e}
\usepackage[svgnames,table]{xcolor}
\usepackage{tikz}
\usepackage{longtable}
\usepackage{changepage}
\usepackage{setspace}
\usepackage{hhline}
\usepackage{multicol}
\usepackage{tabto}
\usepackage{float}
\usepackage{multirow}
\usepackage{makecell}
\usepackage{fancyhdr}
\usepackage[toc,page]{appendix}
\usepackage[hidelinks]{hyperref}
\usetikzlibrary{shapes.symbols,shapes.geometric,shadows,arrows.meta}
\tikzset{>={Latex[width=1.5mm,length=2mm]}}
\usepackage{flowchart}\usepackage[paperheight=11.0in,paperwidth=8.5in,left=1.73in,right=1.73in,top=1.5in,bottom=1.5in,headheight=1in]{geometry}
\usepackage[utf8]{inputenc}
\usepackage[T1]{fontenc}
\TabPositions{0.5in,1.0in,1.5in,2.0in,2.5in,3.0in,3.5in,4.0in,4.5in,5.0in,}

\urlstyle{same}


 %%%%%%%%%%%%  Set Depths for Sections  %%%%%%%%%%%%%%

% 1) Section
% 1.1) SubSection
% 1.1.1) SubSubSection
% 1.1.1.1) Paragraph
% 1.1.1.1.1) Subparagraph


\setcounter{tocdepth}{5}
\setcounter{secnumdepth}{5}


 %%%%%%%%%%%%  Set Depths for Nested Lists created by \begin{enumerate}  %%%%%%%%%%%%%%


\setlistdepth{9}
\renewlist{enumerate}{enumerate}{9}
		\setlist[enumerate,1]{label=\arabic*)}
		\setlist[enumerate,2]{label=\alph*)}
		\setlist[enumerate,3]{label=(\roman*)}
		\setlist[enumerate,4]{label=(\arabic*)}
		\setlist[enumerate,5]{label=(\Alph*)}
		\setlist[enumerate,6]{label=(\Roman*)}
		\setlist[enumerate,7]{label=\arabic*}
		\setlist[enumerate,8]{label=\alph*}
		\setlist[enumerate,9]{label=\roman*}

\renewlist{itemize}{itemize}{9}
		\setlist[itemize]{label=$\cdot$}
		\setlist[itemize,1]{label=\textbullet}
		\setlist[itemize,2]{label=$\circ$}
		\setlist[itemize,3]{label=$\ast$}
		\setlist[itemize,4]{label=$\dagger$}
		\setlist[itemize,5]{label=$\triangleright$}
		\setlist[itemize,6]{label=$\bigstar$}
		\setlist[itemize,7]{label=$\blacklozenge$}
		\setlist[itemize,8]{label=$\prime$}

\setlength{\topsep}{0pt}\setlength{\parskip}{8.04pt}
\setlength{\parindent}{0pt}

 %%%%%%%%%%%%  This sets linespacing (verticle gap between Lines) Default=1 %%%%%%%%%%%%%%


\renewcommand{\arraystretch}{1.3}


%%%%%%%%%%%%%%%%%%%% Document code starts here %%%%%%%%%%%%%%%%%%%%



\begin{document}
\setlength{\parskip}{0.0pt}
{\fontsize{10pt}{12.0pt}\selectfont \textbf{GOLDEN RATIO}\par}\par

{\fontsize{10pt}{12.0pt}\selectfont What makes a single number so interesting that ancient Greeks, Renaissance artists, a 17th century astronomer and a 21st century novelist all would write about it? This $``$golden$"$  number, 1.61803399, represented by the Greek letter Phi, is known as the Golden Ratio. This number has been discovered and rediscovered many times, which is why it has so many names- Golden Number, Golden Proportion, Golden Mean, Golden Section, Divine Proportion and Divine Section.  \par}\par

{\fontsize{10pt}{12.0pt}\selectfont Golden ratio truly is unique in its mathematical properties. It is an irrational number. In mathematics, two quantities are in the \textbf{golden ratio} if their ratio is the same as the ratio of their sum to the larger of the two quantities. For example, it is a special number found by dividing a line into two parts so that the longer part divided by the smaller part is also equal to the whole length divided by the longer part. \par}\par



%%%%%%%%%%%%%%%%%%%% Figure/Image No: 1 starts here %%%%%%%%%%%%%%%%%%%%

\begin{figure}[H]
	\begin{Center}
		\includegraphics[width=2.9in,height=0.88in]{./media/image1.png}
	\end{Center}
\end{figure}


%%%%%%%%%%%%%%%%%%%% Figure/Image No: 1 Ends here %%%%%%%%%%%%%%%%%%%%

\par

{\fontsize{10pt}{12.0pt}\selectfont Part of the \textbf{uniqueness} of Phi is that it can be derived in many other ways than segmenting a line. \par}\par

\begin{itemize}
	\item {\fontsize{10pt}{12.0pt}\selectfont Phi is the only number whose square is greater than itself by one, expressed \href{https://www.goldennumber.net/math/}{mathematically} as $ \Phi $ ² = $ \Phi $  + 1 = 2.618. \par}\par

	\item {\fontsize{10pt}{12.0pt}\selectfont Phi is also the only number whose reciprocal is less than itself by one, expressed as 1/$ \Phi $  = $ \Phi $  – 1 = 0.618. \par}
\end{itemize}\par


\vspace{\baselineskip}
{\fontsize{10pt}{12.0pt}\selectfont We can also derive Phi through Fibonacci sequence. If we go further into the series and we will find that 233/144 = 1.61805, a very close approximation of Phi.\textbf{ Application} of Phi is also found in \href{https://www.goldennumber.net/geometry/}{geometry}, appearing in basic constructions of an equilateral triangle, square and more complex structures like icosahedrons, etc. Phi is more than an obscure term found in mathematics and physics. Golden ratio is so special that it is found everywhere in the world. It also appears in all forms of nature and science. For example: \par}\par

\begin{itemize}
	\item {\fontsize{10pt}{12.0pt}\selectfont Phi extensively appears throughout the \href{https://www.goldennumber.net/life/}{human form}, in the \href{https://www.goldennumber.net/face/}{face}, \href{https://www.goldennumber.net/human-body/}{body}, \href{https://www.goldennumber.net/human-hand-foot/}{fingers}, \href{https://www.goldennumber.net/human-teeth/}{teeth} and even in human \href{https://www.goldennumber.net/dna/}{DNA}, and the impact that this has on the perceptions of human \href{https://www.goldennumber.net/beauty/}{beauty}. For this reason, it is applied in both \href{https://www.goldennumber.net/beauty/}{facial plastic surgery} and \href{https://www.goldennumber.net/human-teeth/}{cosmetic dentistry} to achieve most natural and beautiful results. \par}\par

	\item {\fontsize{10pt}{12.0pt}\selectfont The positions and proportions of the key dimensions of many \href{https://www.goldennumber.net/nature/}{animals} are based on Phi. Examples include the body sections of ants and other insects, the wing dimensions and location of eye-like spots on moths, the \href{https://www.goldennumber.net/spirals/}{spirals} of sea shells and the position of the dorsal fins on porpoises.  \par}\par

	\item {\fontsize{10pt}{12.0pt}\selectfont It’s used as a valuable tool for composition decisions on position and proportion in ANY branch of the design arts, including art, logo design, product design, graphic design, cartoon character design, video layout and composition, fashion design, architecture, photography and much more. Renaissance artists, including Leonardo da Vinci, used it frequently. \par}\par

	\item {\fontsize{10pt}{12.0pt}\selectfont Golden ratio relationships have been found even in the solar system and universe. The \href{https://www.goldennumber.net/solar-system/}{distances of the planets from the sun} correlate surprisingly closely to exponential powers of Phi.  \par}
\end{itemize}\par


\printbibliography
\end{document}